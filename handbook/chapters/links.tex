\chapter{Links}

\section{Lexemes}

Links are made from pairs of lexemes. The name ``lexeme'' was chosen because
links are not associated to words or phrases. A link could be between a word
and a grapheme (such as a hiragana letter or Kanji).

Lexemes don't exist in our database\footnote{This isn't exactly true. We do
keep track of lexeme aliases.}. They would eventually be too numerous to
enumerate comfortably. All possible lexemes are assumed to exist. They act as
nodes in a network of links (in graph theory, the links would be the edges).

This makes Vocabulink rather interesting (although probably not unique) in that
all of the information stored in our database is about connections. We attach
no significance to individual lexemes. They are assumed to exist without
meaning, or at least without a single or universally acceptable meaning. They
are defined entirely by their links to other lexemes (and the information
associated with that link).

Although this may seem obvious or simple, it's a fundamental decision that
shaped the design and philosophy of Vocabulink from a very early stage. The
effects of this decision are not entirely understood yet.

Even though lexemes may be graphemes, they cannot be arbitrary graphics. They
must be representable by a Unicode string.

\section{Association Types}

We can associate lexemes many different ways. The most obvious is for 2 lexemes
that have a direct translation between 2 languages. We'd call this link type a
``translation''. This is what most people think of when they imagine studying
vocabulary from from a foreign language. These are the semantic relationships
between lexemes.

We don't (yet) bother with association types. Other projects such as WordNet
focus on this. Perhaps we'll be able to leverage them someday. But for language
study, I suspect that the benefit is small, especially compared to the work
required to maintain proper associations.

\section{Link Types}

Regardless of the association type of a link, there exists at least 1 way to
remember the link (at least for the links in our database). We refer to this as
the ``link type''. Examples of link types include:

\begin{description}
\item[Association] The simplest of links. This is a general association. It is
  what you're commonly dealing with when studying with flashcards.
\item[Cognate] This is a word that's very similar to its foreign equivalent.
  Cognates usually occur through borrowing. They're generally easy to learn and
  require little if any mnemonic device. Vocabulink currently does not support
  annotating a cognate with additional information, but that may change in the
  future.
\item[Link Word] This well known method of linking words was the basis for the
  creation of Vocabulink (the ``link'' in ``Vocabulink''). The idea is to find
  a native language word or phrase that sounds similar to the foreign lexeme
  and use it for creating a story associating it with the native lexeme.
\item[Relationship] This is an experimental method of linking 2 foreign lexemes
  based on the relationship between 2 native lexemes. Anyone familiar with word
  association used on SAT tests will recognize this method.
\end{description}

\section{Search}

Search box access key: s