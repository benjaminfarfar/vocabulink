\chapter{Links}

\section{Lexemes}

Links are made from pairs of lexemes. The name ``lexeme'' was chosen because
links are not associated to words or phrases. A link could be between a word
and a grapheme (such as a hiragana letter or Kanji).

Lexemes don't exist in our database\footnote{This isn't exactly true. We do
keep track of lexeme aliases.}. They would eventually be too numerous to
enumerate comfortably. All possible lexemes are assumed to exist. They act as
nodes in a network of links (in graph theory, the links would be the edges).

This makes Vocabulink rather interesting (although probably not unique) in that
all of the information stored in our database is about connections. We attach
no significance to individual lexemes. They are assumed to exist without
meaning, or at least without a single or universally acceptable meaning. They
are defined entirely by their links to other lexemes (and the information
associated with that link).

Although this may seem obvious or simple, it's a fundamental decision that
shaped the design and philosophy of Vocabulink from a very early stage. The
effects of this decision are not entirely understood yet.

Even though lexemes may be graphemes, they cannot be arbitrary graphics. They
must be representable by a Unicode string.
