\chapter{Link Review}

Links are rather useless without review.

Vocabulink was actually originally intended to be an ``encyclopedia'' of
vocabulary mnemonics, focusing on linkword mnemonics. Picture Wikipedia with
articles for each word containing images, stories, audio, etc. But a mnemonic,
no matter how powerful, will fade over time without review.

We could leave review up to the student. But there's a lot of work involved in
transferring information between systems. And with mnemonics, what exactly do
you transfer? Do you make flashcards with the native word and a mnemonic on one
side, and on the other side the native word?

By integrating link review into the site, we make it easier to continually
build vocabulary. But perhaps more importantly, we create an objective way to
measure the effectiveness of links.

\section{Adding Links for Review}

When the student indicates that they want to review a link, it's added to their
review list. The link may be one they've established, or one established by
someone else (as long as it's not marked private).

\section{Review Algorithms}

We assist the student by making sure that all of their links for review stay
fresh and current in their mind.

The idea is that a member logs in once a day. They are presented with the links
they need to review. They review them, grade how they did, and the links are
filed for review at a later time. The algorithm remains a black box.

\subsection{SM2}

Vocabulink currently only has 1 algorithm:
\href{http://www.supermemo.com/english/ol/sm2.htm}{SM2}. This is not the best
we can get, but it's a relatively simple algorithm. Future SuperMemo algorithms
require a database of review statistics before they can be implemented, so this
is a good starting point.

We may diverge from SuperMemo algorithms once we've had a chance to study a
significant amount of review data.
